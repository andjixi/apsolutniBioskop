\documentclass[a4paper]{article}
\usepackage{fullpage}
\usepackage{graphicx}
\usepackage{amsmath}
\usepackage[serbian]{babel}
\usepackage[T1]{fontenc}
\usepackage[utf8]{inputenc}
\usepackage{lmodern}
\usepackage{tabularx}
\usepackage{float}

\title{Informacioni sistem bioskopa: Apsolutni bioskop\\ 
\small{Dokumentacija u okviru projekta iz predmeta\\ Informacioni sistemi\\ Matematički fakultet}}
\author{Marko Paunović, Bogdan Stojadinović , Anđela Jovanović \\ mi251039@alas.matf.bg.ac.rs \\ mi251028@alas.matf.bg.ac.rs \\ mi251035@alas.matf.bg.ac.rs\\}
\date{Januar 2026.}

\begin{document}

\maketitle

\tableofcontents

\newpage

\section{Analiza sistema}

\subsection{O čemu se radi u ovom projektu?}
Ovaj rad je nastao kao deo projektnog zadatka na predmetu ``Informacioni sistemi'' na master studijama Matematičkog fakulteta. Glavna ideja bila je da teoriju o planiranju sistema primenimo u praksi i pokažemo kako se jedan poslovni proces može detaljno isplanirati.

Napravili smo plan za informacioni sistem koji pomaže firmi da lakše prodaje svoje usluge i upravlja zaposlenima. Najviše smo se fokusirali na to kako da automatizujemo svakodnevne poslove (poput prodaje na blagajni) i kako da osiguramo da svako u sistemu vidi samo ono što mu je dozvoljeno. Da bismo sistem prikazali sa svih strana, koristili smo različite dijagrame: od opštih Use Case dijagrama, preko detaljnih dijagrama aktivnosti i sekvenci, pa sve do BPMN dijagrama koji pokazuju saradnju između ljudi i softvera.

\subsection{Ko sve koristi sistem?}
U našem sistemu postoje četiri glavne grupe korisnika, a svaka od njih ima svoju specifičnu ulogu:
\begin{enumerate}
    \item \textbf{Korisnik} (običan kupac ili klijent)
    \item \textbf{Blagajnik} (osoba koja radi na prodajnom mestu)
    \item \textbf{Menadžer} (osoba zadužena za tim i organizaciju)
    \item \textbf{Sistem administrator} (tehnička podrška i kontrola sistema)
\end{enumerate}

\subsection{Šta korisnici zapravo mogu da rade?}
Evo kratkog pregleda onoga što je svakoj grupi korisnika omogućeno u sistemu:
\begin{enumerate}
\item \textbf{Korisnik} \\
On je tu da bi koristio usluge. Prvo se registruje i prijavljuje, a zatim može da pregleda ponudu, kupuje ili rezerviše ono što mu treba. Sistem brine o tome da su njegovi podaci sigurni i da niko drugi ne može da vidi njegovu istoriju kupovine ili lične informacije.

\item \textbf{Blagajnik} \\
Njegov posao je rad sa ljudima i kucanje računa. On proverava da li je neka usluga dostupna, vrši naplatu i izdaje potvrde korisnicima. Blagajnik vidi sve što je vezano za prodaju, ali nema pristup podacima o platama ili podešavanjima samog sistema.

\item \textbf{Menadžer} \\
On vodi računa o tome da sve funkcioniše kako treba. On ne kuca u sistem podatke o prodaji, već prati izveštaje. Kada se pojavi potreba za novim radnikom, on je taj koji šalje zvaničan zahtev administratoru. Njegov glavni zadatak u ovom modelu je da odobri i pokrene proces zapošljavanja.

\item \textbf{Sistem administrator} \\
On je najbitniji za sistem u tehničkom pogledu. On kreira naloge za nove zaposlene tek kada dobije zahtev od menadžera. Takođe, on brine o bezbednosti, to jest ako neko zaboravi lozinku ili mu nalog bude blokiran zbog previše pogrešnih pokušaja, administrator je tu da reši problem i vrati sistem u normalu.
\end{enumerate}

\vspace{5pt}

\newpage
\subsection{Dijagram slučajeva upotrebe}
Ovaj dijagram prikazuje sve grupe korisnika i funkcije koje su im dostupne unutar sistema. On služi da se na početku jasno definišu granice sistema i šta svaka uloga može da uradi.
\begin{figure}[h]
    \centering
    \includegraphics[width=1\textwidth]{dijagram_slucajeva_upotrebe.jpg}
    \caption{Dijagram slučajeva upotrebe}
    \label{fig:dijagram_slucajeva_upotrebe}
\end{figure}

\newpage
\section{Dijagrami toka podataka}
Dijagrami toka podataka (DFD) opisuju kako se informacije kreću kroz sistem, od samog unosa do njihovog skladištenja u bazi.

\subsection{DFD dijagram konteksta}
Dijagram konteksta prikazuje sistem kao jednu celinu u odnosu na okruženje. Na njemu se vidi samo razmena podataka sa korisnicima i bazom podataka, bez detalja o samoj obradi unutar softvera.

\begin{figure}[h]
    \centering
    \includegraphics[width=1\textwidth]{dfd_dijagram_konteksta.jpg}
    \caption{DFD dijagram konteksta}
    \label{fig:dfd_dijagram_konteksta}
\end{figure}

\newpage
\subsection{DFD dijagram nivoa 0}
\begin{figure}[h]
    \centering
    \includegraphics[width=0.935\textwidth]{dfd_dijagram_nivoa_0.jpg}
    \caption{DFD dijagram nivoa 0}
    \label{fig:dfd_dijagram_nivoa_0}
\end{figure}


\newpage
\section{Slučajevi upotrebe}
\documentclass{article}
\usepackage{graphicx}
\usepackage[serbian]{babel}

\begin{document}

\section*{Slučaj upotrebe: Kupovina i registracija karte online}
\label{uc:online_kupovina_i_registracija}
\vspace{10pt}

\begin{itemize}
    \item \textbf{Kratak opis:} Korisnik pristupa aplikaciji kako bi pogledao repertoar bioskopa, odabrao film koji želi da pogleda i kupi ili rezerviše kartu za odredjeni termin.
    
    \item \textbf{Akteri:} 
    \begin{itemize}
        \item \textbf{Korisnik} - osoba koja koristi aplikaciju za kupovinu ili rezervaciju karata.
    \end{itemize}
    \vspace{5pt}

    \item \textbf{Preduslovi:} 
    \begin{itemize}
        \item Korisnik je povezan na internet, registrovan i ulogovan je u aplikaciju.
        \item Sistem je povezan sa bazom podataka.
    \end{itemize}
    \vspace{5pt}
    
    \item \textbf{Postuslovi:} 
    \begin{itemize}
        \item Korisnik je uspešno kupio ili rezervisao kartu za izabrani film i termin.
    \end{itemize}
    \vspace{5pt}
    
    \item \textbf{Osnovni tok:}
    \begin{enumerate}
        \item Korisnik pristupa aplikaciji.
        \item Sistem prikazuje početnu stranicu sa opcijom za pregled repertoara bioskopa.
        \item Korisnik bira opciju za pregled repertoara.
        \item Sistem prikazuje listu dostupnih filmova sa terminima projekcija.
        \item Korisnik bira film koji želi da pogleda.
        \item Sistem prikazuje detalje o izabranom filmu, uključujući dostupne termine i cene karata.
        \item Korisnik bira željeni termin projekcije.
        \item Sistem prikazuje raspoloživa sedišta za izabrani termin.
        \item Korisnik bira sedište koje želi da rezerviše ili kupi.
        \item Sistem prikazuje opciju za kupovinu ili rezervaciju karte.
            \begin{itemize}
                \item Ukoliko je korisnik izabrao kupovinu izvršava se podtok P1
                \item Ukoliko je korisnik izabrao rezervaciju izvršava se podtok P2
            \end{itemize}
        \item Sistem šalje potvrdu o uspešnosti.
    \end{enumerate}

    \item \textbf{Podtokovi:}
    
    P1: Kupovina karte
    \begin{enumerate}
        \item Sistem proverava bonus poene korisnika i nudi popust ukoliko korisnik ima dovoljan broj poena na nalogu.
        \item Sistem prikazuje konačnu cenu i opciju za plaćanje.
        \item Korisnik unosi podatke o plaćanju.
        \item Sistem obradjuje plaćanje.
        \item Sistem ažurira raspoloživost sedišta u bazi podataka.
        \item Sistem generiše elektronsku kartu i šalje je korisniku putem e-pošte i prikazuje na ekranu.
        \item Sistem ažurira bonus poene korisnika na nalogu.
    \end{enumerate}

    P2: Rezervacija karte
    \begin{enumerate}
        \item Korisnik potvrdjuje rezervaciju.
        \item Sistem obradjuje rezervaciju i čuva podatke o rezervisanom sedištu.
        \item Sistem ažurira raspoloživost sedišta u bazi podataka.
        \item Sistem šalje potvrdu o rezervaciji korisniku putem e-pošte i prikazuje na ekranu.
    \end{enumerate}
    \vspace{5pt}
    
    \item \textbf{Alternativni tokovi:}
    \begin{itemize}
        \item \textbf{A1: Korisnik odustaje od procesa:} U bilo kom momentu korisnik može prekinuti proces i sistem poništava rezervaciju pre potvrde.
        \item \textbf{A2: Neuspelo plaćanje:} Ukoliko plaćanje ne uspe, sistem obaveštava korisnika i nudi opciju da pokuša ponovo ili odustane.
    \end{itemize}
    \vspace{5pt}

    \item \textbf{Specijalni zahtevi:} /
    \vspace{5pt}
    
    \item \textbf{Dodatne informacije:} /
    \vspace{5pt}
\end{itemize}

\begin{figure}[h]
    \centering
    \includegraphics[width=0.75\textwidth]{dijagram_aktivnosti_online_kupovina.jpg}
    \caption{Dijagram aktivnosti: Online kupovina i registracija karte}
    \label{fig:my_image_label}
\end{figure}

\end{document}
\newpage
\documentclass{article}
\usepackage{graphicx}

\begin{document}

\section*{Slučaj upotrebe: Kupovina na blagajni}
\label{uc:kupovina_na_blagajni}
\vspace{10pt}

\begin{itemize}
    \item \textbf{Kratak opis:} Korisnik dolazi na blagajnu bioskopa kako bi kupio kartu, preuzeo prethodno rezervisanu kartu. Nakon kupljene karte, ili u slučaju gde korisnik već poseduje kartu, blagajnik korisniku nudi kupovinu grickalica i pića. Sistem evidentira sve obavljene transakcije i ažurira podatke u bazi.
    
    \item \textbf{Akteri:} 
    \begin{itemize}
        \item \textbf{Korisnik} - osoba koja dolazi na blagajnu.
        \item \textbf{Blagajnik} - zaposleni koji obavlja prodaju i interakciju sa korisnikom.
    \end{itemize}
    \vspace{5pt}

    \item \textbf{Preduslovi:} 
    \begin{itemize}
        \item Bioskop je otvoren i blagajna je u funkciji.
        \item Korisnik je fizički prisutan na blagajni i bar jedan blagajnik je dostupan.
        \item Sistem je povezan sa bazom podataka o projekcijama, cenama, raspoloživim sedištima, grickalicama i pićima. 
        \item Blagajnik je ulogovan u sistem.
    \end{itemize}
    \vspace{5pt}
    
    \item \textbf{Postuslovi:} 
    \begin{itemize}
        \item Korisnik je uspešno izvršio transakciju zbog koje je došao na blagajnu, a sistem je ažurirao sve relevantne podatke u bazi.
    \end{itemize}
    \vspace{5pt}
    
    \item \textbf{Osnovni tok:}
    \begin{enumerate}
        \item Korisnik prilazi slobodnom blagajniku.
        \item Korisnik bira jednu od opcija:
            \begin{itemize}
                \item Ukoliko želi da kupi kartu, prelazi se na podtok P1.
                \item Ukoliko želi da preuzme prethodno rezervisanu kartu, prelazi se na podtok P2.
                \item Ukoliko ne želi ni jedno ni drugo, prelazi se na korak 3.
            \end{itemize}
        \item Blagajnik pita korisnika da li želi da kupi grickalice ili piće.
            \begin{itemize}
                \item Ukoliko korisnik ne želi ništa, prelazi se na korak 6.
            \end{itemize}
        \item Korisnik bira grickalice i/ili piće.
        \item Blagajnik unosi izabrane artikle u porudžbinu.
        \item Sistem prikazuje ukupnu cenu za izabrane artikle.
        \item Blagajnik traži od korisnika način plaćanja.
        \item Korisnik bira način plaćanja i daje potrebne podatke.
        \item Sistem obradjuje plaćanje.
    \end{enumerate}

    \item \textbf{Podtokovi:}
    
    P1: Kupovina karte
    \begin{enumerate}
        \item Blagajnik pita korisnika za film i termin koji želi da pogleda. 
        \item Blagajnik zahteva podatke o traženom filmu i terminu. 
        \item Sistem prikazuje dostupna sedišta za izabrani termin. 
        \item Korisnik bira sedište/a koje želi da kupi. 
        \item Blagajnik unosi izabrana sedišta u sistem. 
        \item Blagajnik unosi karte u porudžbinu. 
        \item Prelazi se na korak 4 osnovnog toka.
    \end{enumerate}

    P2: Preuzimanje rezervisane karte
    \begin{enumerate}
        \item Blagajnik traži od korisnika podatke o rezervaciji (npr. broj rezervacije, ime). 
        \item Sistem pronalazi rezervaciju i prikazuje detalje. 
        \item Blagajnik proverava validnost rezervacije.
            \begin{itemize}
                \item Ukoliko je rezervacija validna prelazi se na korak 4 osnovnog toka.
                \item Ukoliko rezervacija nije validna, blagajnik obaveštava korisnika i vraća se na korak 2 osnovnog toka. 
            \end{itemize}
        \item Blagajnik potvrdjuje preuzimanje karte i unosi karte u porudžbinu. 
        \item Prelazi se na korak 4 osnovnog toka.
    \end{enumerate}
    \vspace{5pt}
    
    \item \textbf{Alternativni tokovi:}
    \begin{itemize}
        \item \textbf{A1: Nema slobodnih sedišta:} Ukoliko nema slobodnih sedišta za izabrani termin, sistem obaveštava blagajnika koji zatim obaveštava korisnika i nudi alternativne termine ili filmove i prelazi na korak 1 podtoka P1.
        \item \textbf{A2: Korisnik odustaje od procesa:} U bilo kom momentu korisnik može odlučiti da ne nastavi sa kupovinom ili preuzimanjem karte, i blagajnik zatim završava interakciju.
        \item \textbf{A3: Neuspelo plaćanje:} Ukoliko plaćanje ne uspe, sistem obaveštava blagajnika koji zatim obaveštava korisnika i nudi opciju da pokuša ponovo ili odustane.
        \item \textbf{A4: Nedostupnost artikala (grickalice i piće):} Ukoliko izabrani artikli nisu dostupni, blagajnik obaveštava korisnika i nudi alternativne artikle ili završava proces bez kupovine artikala.
        \item \textbf{A5: Odustajanje od artikala:} Korisnik u bilo kom momentu može da odluči da ne kupi neki od artikala iz porudžbine pre nego što se predje na plaćanje, i blagajnik zatim ažurira porudžbinu u skladu sa tim.
    \end{itemize}
    \vspace{5pt}

    \item \textbf{Specijalni zahtevi:} /
    \vspace{5pt}
    
    \item \textbf{Dodatne informacije:} /
    \vspace{5pt}
\end{itemize}

\end{document}
\newpage
\subsection{Slučaj upotrebe: Registracija}
\label{uc:registracija}
\vspace{10pt}

\begin{itemize}
    \item \textbf{Kratak opis:} Korisnik kreira novi nalog u sistemu kako bi dobio pristup funkcionalnostima kao što su rezervacija karata i sakupljanje bonus poena.
    
    \item \textbf{Akteri:} 
    \begin{itemize}
        \item Korisnik (Gost)
    \end{itemize}
    \vspace{5pt}

    \item \textbf{Preduslovi:} 
    \begin{itemize}
        \item Korisnik ima pristup internetu i aplikaciji.
        \item Korisnik nije ulogovan.
    \end{itemize}
    \vspace{5pt}
    
    \item \textbf{Postuslovi:} 
    \begin{itemize}
        \item Novi korisnički nalog je uspešno kreiran, aktiviran i sačuvan u bazi podataka.
        \item Korisnik je spreman za prijavu.
    \end{itemize}
    \vspace{5pt}
    
    \item \textbf{Osnovni tok:}
    \begin{enumerate}
        \item Korisnik bira opciju za registraciju na početnoj stranici aplikacije.
        \item Sistem prikazuje formu za unos podataka (Ime, Prezime, Email, Lozinka, Broj telefona).
        \item Korisnik unosi tražene podatke i potvrđuje unos. 
        \item Sistem validira format unetih podataka (email, jačina lozinke).
        \item Sistem proverava da li korisnik sa unetim email-om već postoji u bazi.
        \item Sistem upisuje podatke o novom nalogu u bazu podataka sa statusom 'neaktivan'.
        \item Sistem šalje potvrdni email (verifikacioni link) na unetu adresu.
        \item Sistem obaveštava korisnika da proveri svoj email.
        \item Korisnik pristupa svom email-u i klikće na verifikacioni link.
        \item Sistem pronalazi nalog u bazi, menja status naloga na 'aktivan' i trajno beleži vreme aktivacije.
        \item Sistem obaveštava korisnika o uspešnoj registraciji i aktivaciji.
    \end{enumerate}

    \item \textbf{Podtokovi:} /
    \vspace{5pt}
    
    \item \textbf{Alternativni tokovi:}
    \begin{itemize}
        \item \textbf{A1: Nevalidni podaci:} Sistem prikazuje grešku o neispravnom formatu podataka i traži od korisnika da ih ispravi. Korisnik ispravlja podatke i tok se vraća na korak 3 osnovnog toka.
        \item \textbf{A2: Korisnik već postoji:} Sistem obaveštava korisnika da nalog sa tim email-om već postoji. Sistem nudi opciju za prelazak na prijavu (Login) ili reset lozinke.
        \item \textbf{A3: Odustajanje:} Korisnik u bilo kom trenutku može odustati od registracije povratkom na početnu stranu.
    \end{itemize}
    \vspace{5pt}

    \item \textbf{Specijalni zahtevi:}
    \begin{itemize}
        \item Lozinka mora sadržati minimum 8 karaktera, jedno veliko slovo i jedan broj.
        \item Verifikacioni link važi 24 sata.
    \end{itemize}
    \vspace{5pt}
    
    \item \textbf{Dodatne informacije:} /
    \vspace{5pt}
\end{itemize}

\newpage
\subsubsection*{Dijagram aktivnosti}
\begin{figure}[H]
\centering
\makebox[\textwidth][c]{%
  \includegraphics[width=1.2\textwidth]{../dijagrami/aktivnosti/registracija_dijagram_aktivnosti.jpg}
}
\caption{Dijagram aktivnosti za registraciju korisnika}
\label{fig:aktivnost-registracija}
\end{figure}

\newpage
\subsubsection*{Dijagram sekvence}
\begin{figure}[H]
\centering
\makebox[\textwidth][c]{%
  \includegraphics[width=1.2\textwidth]{../dijagrami/sekvence/registracija_dijagram_sekvence.jpg}
}
    \caption{Dijagram sekvence za registraciju korisnika}
    \label{fig:sekvenca-registracija}
\end{figure}

\newpage
\subsubsection*{Dijagram stanja Naloga}
\begin{figure}[H]
\centering
\makebox[\textwidth][c]{%
  \includegraphics[width=0.5\textwidth]{../dijagrami/stanja/dijagram_stanja_naloga_registracija.jpg}
}
    \caption{Dijagram stanja naloga}
    \label{fig:stanje-registracija}
\end{figure}

\newpage
\subsection{Slučaj upotrebe: Autorizacija}
\label{uc:autorizacija}
\vspace{10pt}

\begin{itemize}
    \item \textbf{Kratak opis:} Proces identifikacije i provere prava pristupa aktera (korisnika ili zaposlenog) kako bi pristupili zaštićenim delovima sistema u skladu sa svojom ulogom.
\vspace{10pt}
    
    \item \textbf{Akteri:} 
    \begin{itemize}
        \item Korisnik
        \item Menadžer
        \item Sistem administrator
        \item Blagajnik
    \end{itemize}
    \vspace{5pt}
    
    \item \textbf{Preduslovi:} 
    \begin{itemize}
        \item Akter poseduje validan nalog u sistemu.
        \item Aplikacija je dostupna.
    \end{itemize}
    \vspace{5pt}
    
    \item \textbf{Postuslovi:} 
    \begin{itemize}
        \item Akter je uspešno ulogovan i sistem mu prikazuje odgovarajući meni u zavisnosti od uloge.
    \end{itemize}
    \vspace{5pt}
    
    \item \textbf{Osnovni tok:}
    \begin{enumerate}
        \item Akter bira opciju za prijavu (Login).
        \item Sistem prikazuje formu za unos kredencijala (Korisničko ime/Email i Lozinka).
        \item Akter unosi svoje podatke i potvrđuje prijavu.
        \item Sistem pristupa bazi podataka, proverava ispravnost lozinke.
        \item Sistem proverava status naloga aktera (aktivan, blokiran, suspendovan).
        \item Sistem utvrđuje ulogu aktera (Korisnik, Menadžer, Administrator, Blagajnik).
        \item Sistem ažurira polje 'Poslednje prijavljivanje' (\textit{Last Login Time}) u bazi podataka za taj nalog.
        \item Sistem odobrava pristup i preusmerava aktera na odgovarajuću početnu stranicu/kontrolnu tablu.
    \end{enumerate}
    
    \item \textbf{Podtokovi:} /
    \vspace{5pt}
    
    \item \textbf{Alternativni tokovi:}
    \begin{itemize}
        \item \textbf{A1: Pogrešni podaci:} Sistem obaveštava aktera o neuspešnoj prijavi. Akteru se omogućava ponovni pokušaj.
        \item \textbf{A2: Blokiran nalog:} Sistem prikazuje poruku o tome da je nalog neaktivan/blokiran i upućuje aktera na kontakt podrške ili povratak na početnu stranu.
    \end{itemize}
    \vspace{5pt}
    
    \item \textbf{Specijalni zahtevi:}
    \begin{itemize}
        \item Sistem beleži pokušaje prijavljivanja i privremeno blokira nalog nakon pet neuspelih pokušaja.
    \end{itemize}
    \vspace{5pt}
    
    \item \textbf{Dodatne informacije:} /
    \vspace{5pt}
\end{itemize}

\newpage

\begin{figure}[H]
\subsubsection*{Dijagram aktivnosti}
\centering
\makebox[\textwidth][c]{%
  \includegraphics[width=1.2\textwidth]{Autorizacija/autorizacija_dijagram_aktivnosti.jpg}
}
    \caption{Dijagram aktivnosti naloga  za autorizaciju korisnika}
    \label{fig:aktivnost-autorizacija}
\end{figure}

\newpage
\subsubsection*{Dijagram sekvence}
\begin{figure}[H]
\centering
\makebox[\textwidth][c]{%
  \includegraphics[width=1.2\textwidth]{Autorizacija/autorizacija_dijagram_sekvence.jpg}
}
    \caption{Dijagram sekvence naloga  za autorizaciju korisnika}
    \label{fig:sekvenca-autorizacija}
\end{figure}

\newpage
\subsubsection*{Dijagram stanja Naloga}

\begin{figure}[H]
\centering
\makebox[\textwidth][c]{%
  \includegraphics[width=1\textwidth]{Autorizacija/dijagram_stanja_naloga.jpg}
}
    \caption{Dijagram stanja naloga}
    \label{fig:stanje-autorizacija}
\end{figure}

\newpage

\section*{Slučaj upotrebe: Upravljanje podacima}
\vspace{10pt}

\begin{itemize}
    \item \textbf{Kratak opis:} Sistem Administrator dodaje ili briše podatke iz baze.
    
    \item \textbf{Akteri:} 
    \begin{itemize}
        \item \textbf{Sistem Administrator} - osoba koja upravlja sistemom.
    \end{itemize}

    \item \textbf{Preduslovi:} 
    \begin{itemize}
        \item Sistem Administrator je povezan na internet, registrovan i ulogovan u aplikaciju kao Sistem Administrator.
        \item Sistem je povezan sa bazom podataka.
    \end{itemize}
    
    \item \textbf{Postuslovi:} 
    \begin{itemize}
        \item Sistem Administrator je uspešno izmenio podatke.
    \end{itemize}
    
    \item \textbf{Osnovni tok:}
    \begin{enumerate}
        \item Sistem Administrator pristupa aplikaciji.
        \item Sistem Administrator pritiska dugme "Upravljaj podacima".
        \item Sistem prikazuje prazno polje za unos teksta i dugme "Izvrši".
        \item Sistem Administrator u polje unosi SQL naredbu.
        \item Sistem Administrator pritiska dugme "Izvrši".
        \item Sistem izvršava SQL naredbu.
        \item Sistem šalje potvrdu uspešnosti.
    \end{enumerate}

    \item \textbf{Podtokovi:} /
    \item \textbf{Alternativni tokovi:} /
    \item \textbf{Specijalni zahtevi:} /
    \item \textbf{Dodatne informacije:} /
\end{itemize}

\newpage
\subsection*{Dijagram aktivnosti}

\begin{figure}[h!]
\centering
\makebox[\textwidth][c]{%
  \includegraphics[width=1.5\textwidth]{Upravljanje Podacima/upravljanje_podacima_dijagram_aktivnosti.jpg}
}
    \caption{Dijagram aktivnosti slučaja upotrebe Upravljanje podacima}
    \label{fig:aktivnost-autorizacija}
\end{figure}

\newpage
\subsection*{Dijagram sekvence}

\begin{figure}[h!]
\centering
\makebox[\textwidth][c]{%
  \includegraphics[width=1.5\textwidth]{Upravljanje Podacima/upravljanje_podacima_dijagram_sekvence.jpg}
}
    \caption{Dijagram sekvence slučaja upotrebe Upravljanje podacima}
    \label{fig:aktivnost-autorizacija}
\end{figure}

\newpage
\subsection*{Dijagram stanja}

\begin{figure}[h!]
\centering
\makebox[\textwidth][c]{%
  \includegraphics[width=1\textwidth]{Upravljanje Podacima/dijagram_stanja_filma.jpg}
}
    \caption{Dijagram stanja filma}
    \label{fig:aktivnost-autorizacija}
\end{figure}

\newpage
\documentclass{article}
\usepackage{graphicx} 

\begin{document}

\section*{Slučaj upotrebe: Dodavanje novih zaposlenih}
\label{uc:dodavanje_zaposlenih}
\vspace{10pt}

\begin{itemize}
    \item \textbf{Kratak opis:} Na osnovu zahteva od strane Menadžera, Sistem administrator unosi podatke o novom zaposlenom u sistem, dodeljujući mu odgovarajuća prava pristupa.
    \vspace{5pt}
    
    \item \textbf{Akteri:} 
    \begin{itemize}
        \item Sistem administrator
        \item Menadžer
    \end{itemize}
    \vspace{5pt}
    
    \item \textbf{Preduslovi:} 
    \begin{itemize}
        \item Sistem administrator je ulogovan u sistem i ima privilegije za upravljanje korisnicima.
        \item Menadžer je dostavio sve potrebne podatke o novom zaposlenom (Ime, Uloga, Kontakt, itd.).
    \end{itemize}
    \vspace{5pt}
    
    \item \textbf{Postuslovi:} 
    \begin{itemize}
        \item Novi nalog za zaposlenog je kreiran sa dodeljenom ulogom.
        \item Podaci su trajno upisani u bazu podataka.
        \item Zaposleni je obavešten o početnim kredencijalima.
    \end{itemize}
    \vspace{5pt}
    
    \item \textbf{Osnovni tok:}
    \begin{enumerate}
        \item Sistem administrator prima zvanični zahtev (dokument ili digitalnu formu) od Menadžera za kreiranje naloga novog zaposlenog.
        \item Sistem administrator pristupa sekciji za upravljanje zaposlenima (Administrativni panel).
        \item Administrator bira opciju "Dodaj novog zaposlenog".
        \item Sistem prikazuje formu za unos podataka.
        \item Administrator unosi sve potrebne podatke (Ime, Prezime, JMBG, Email, Uloga, Početna lozinka) na osnovu zahteva Menadžera.
        \item Sistem validira ispravnost formata unetih podataka.
        \item Sistem proverava u bazi podataka da li već postoji nalog sa unetim JMBG-om ili Email-om.
        \item Sistem proverava da li Administrator ima dozvolu (autorizaciju) da dodeli izabranu ulogu (npr. kreiranje drugog Menadžera ili Admina).
        \item Sistem trajno upisuje podatke o novom zaposlenom, početnoj lozinci i dodeljenoj ulozi u bazu podataka.
        \item Sistem generiše potvrdu o uspešnom kreiranju naloga i šalje početne kredencijale novom zaposlenom putem emaila.
        \item Sistem obaveštava Menadžera da je nalog uspešno kreiran i spreman za upotrebu.
    \end{enumerate}

    \item \textbf{Podtokovi:} /
    \vspace{5pt}
    
    \item \textbf{Alternativni tokovi:}
    \begin{itemize}
        \item \textbf{A1: Nevalidni podaci:} Sistem obaveštava Administratora o grešci u formatu (npr. neispravan email). Administrator mora da ispravi podatke ili da traži ispravku od Menadžera.
        \item \textbf{A2: Zaposleni već postoji:} Sistem obaveštava Administratora da je prijava nevažeća zbog dupliranih identifikatora (JMBG/Email). Administrator obaveštava Menadžera.
        \item \textbf{A3: Nedovoljne privilegije:} Sistem odbija kreiranje naloga sa izabranom ulogom.
    \end{itemize}
    \vspace{5pt}
    
    \item \textbf{Specijalni zahtevi:}
    \begin{itemize}
        \item Sistem mora da zahteva promenu lozinke prilikom prvog logovanja novog zaposlenog.
        \item JMBG se mora koristiti kao jedinstveni identifikator prilikom provere duplikata.
    \end{itemize}
    \vspace{5pt}
    
    \item \textbf{Dodatne informacije:} /
    \vspace{5pt}
\end{itemize}

\newpage
\subsection*{Dijagram aktivnosti}
\begin{figure}[h!]
\centering
\makebox[\textwidth][c]{%
  \includegraphics[width=1.5\textwidth]{dodavanje_novih_zaposlenih_dijagaram_aktivnosti.jpg}
}
\caption{Dijagram aktivnosti za dodavanje novog zaposlenog}
\label{fig:aktivnost-radnici}
\end{figure}

\newpage
\subsection*{Dijagram sekvence}

\begin{figure}[h!]
\centering
\makebox[\textwidth][c]{%
  \includegraphics[width=1.5\textwidth]{dodavanje_novih_zaposlenih_dijagram_sekvence.jpg}
}
    \caption{Dijagram sekvence za dodavanje novog zaposlenog}
    \label{fig:sekvenca-radnici}
\end{figure}

\newpage
\subsection*{BPMN Dijagram saradnje za dodavanje novog zaposlenog}

\begin{figure}[h!]
\centering
\makebox[\textwidth][c]{%
  \includegraphics[width=1.5\textwidth]{dodavanje_zaposlenog_bpmn_saradnje.jpg}
}
    \caption{BPMN Dijagram saradnje za dodavanje novog zaposlenog}
    \label{fig:bpmn}
\end{figure}

\end{document}

\newpage
\subsection{Slučaj upotrebe: Izmena repertoara}
\vspace{10pt}

\begin{itemize}
    \item \textbf{Kratak opis:} Menadžer pristupa aplikaciji kako bi izmenio repertoar filmova za svoj bioskop za sledeću nedelju. U raspored može da ubaci i nove filmove (koji su tek izašli ili su tek stigli u bioskop).
Sistem Administrator dodaje u sistem filmove koji su tek izašli ili briše one koji su zastareli.
    
    \item \textbf{Akteri:} 
    \begin{itemize}
        \item \textbf{Menadžer} - osoba koja upravlja bioskopom.
        \item \textbf{Sistem Administrator} - osoba koja upravlja sistemom.
    \end{itemize}

    \item \textbf{Preduslovi:} 
    \begin{itemize}
        \item Menadžer je povezan na internet, registrovan i ulogovan u aplikaciju kao menadžer.
        \item Sistem Administrator je povezan na internet, registrovan i ulogovan u aplikaciju kao Sistem Administrator.
        \item Sistem je povezan sa bazom podataka.
    \end{itemize}
    
    \item \textbf{Postuslovi:} 
    \begin{itemize}
        \item Menadžer je uspešno podesio repertoar za sledeću nedelju.
        \item Sistem Administrator je uspešno dodao ili obrisao film.
    \end{itemize}
    
    \item \textbf{Osnovni tok:}
    \begin{enumerate}
        \item Sistem Administrator pristupa aplikaciji.
        \item Sistem Administrator pritiska dugme "Izmeni dostupne filmove".
            \begin{itemize}
                \item Ukoliko pritisne "Dodaj film" izvršava se podtok P1.
                \item Ukoliko pritisne "Ukloni film" izvršava se podtok P2.
            \end{itemize}
        \item Sistem šalje potvrdu uspešnosti.
        \item Menadžer pristupa aplikaciji.
        \item Pritiska dugme "Izmeni repertoar".
        \item Sistem prikazuje listu termina sa znakom "+" pored svakog termina.
        \item Menadžer pritiska taster "+" pored termina koji želi da izmeni.
        \item Sistem otvara listu dostupnih filmova.
        \item Menadžer bira film koji će se puštati u tom terminu.
        \item Sistem vraća Menadžera na listu termina i umesto znaka "+" prikazuje naziv izabranog filma.
        \item Kada Menadžer završi sa izmenama repertoara pritiska dugme "Sačuvaj".
        \item Sistem šalje potvrdu uspešnosti.
    \end{enumerate}

    \item \textbf{Podtokovi:}
    
    P1: Dodavanje filma
    \begin{enumerate}
        \item Sistem Administrator unosi podatke o filmu koji želi da doda.
        \item Sistem Administrator pritiska dugme "Sačuvaj".
    \end{enumerate}

    P2: Uklanjanje filma
    \begin{enumerate}
        \item Sistem Administrator bira film koji želi da ukloni.
        \item Sistem Administrator pritiska dugme "Ukloni".
    \end{enumerate}
    
    \item \textbf{Alternativni tokovi:}
    \begin{itemize}
        \item A1: Menadžer odustaje od procesa: - U bilo kom momentu može prekinuti proces, a sistem poništava izmene.
        \item A2: Sistem Administrator odustaje od procesa: - U bilo kom momentu može prekinuti proces, a sistem poništava izmene.
    \end{itemize}

    \item \textbf{Specijalni zahtevi:} /
    
    \item \textbf{Dodatne informacije:} /
\end{itemize}

\subsection*{Dijagram aktivnosti}
\begin{figure}[H]
\centering
\makebox[\textwidth][c]{%
  \includegraphics[width=0.73\textwidth]{../dijagrami/aktivnosti/izmena_repertoara_dijagram_aktivnosti.jpg}
}
    \caption{Dijagram aktivnosti slučaja upotrebe Izmena repertoara}
    \label{fig:aktivnost-izmena-repertoara}
\end{figure}

\newpage
\subsubsection*{Dijagrami sekvence}
\begin{figure}[H]
\centering
\makebox[\textwidth][c]{%
  \includegraphics[width=1.2\textwidth]{../dijagrami/sekvence/izmena_repertoara_upravljanje_filmovima_dijagram_sekvence.jpg}
}
    \caption{Dijagram sekvence za upravljanje filmovima}
    \label{fig:sekvenca-upravljanje-filmovima}
\end{figure}

\newpage
\begin{figure}[H]
\centering
\makebox[\textwidth][c]{%
  \includegraphics[width=1\textwidth]{../dijagrami/sekvence/izmena_repertoara_dijagram_sekvence.jpg}
}
    \caption{Dijagram sekvence za izmenu repertoara}
    \label{fig:sekvena-izmena-repertoara}
\end{figure}

\newpage
\subsubsection*{BPMN dijagram}
\begin{figure}[H]
\centering
\makebox[\textwidth][c]{%
  \includegraphics[width=1.3\textwidth]{../dijagrami/bpmn/izmena_repertoara_bpmn_dijagram.jpg}
}
    \caption{BPMN dijagram slučaja upotrebe Izmena repertoara}
    \label{fig:bpmn-izmena-repertoara}
\end{figure}




\newpage
\section{Detaljna specifikacija modela baze podataka}

Na osnovu slučaja upotrebe osmišljena je sledeća struktura baze podataka definišući klase:
\begin{itemize}
    \item Korisnik
    \item Karta
    \item Projekcija
    \item Film
    \item Bioskop
    \item Sala
    \item Sedište
    \item Zaposlen (Sistem Administrator, Menadžer, Blagajnik)
    \item Slobodna Sedišta
\end{itemize}

\begin{figure}[h!]
    \centering
    \makebox[\textwidth][c]{
        \includegraphics[width=1.2\textwidth]{dijagram_klasa.jpg}
        }
        \caption{Dijagram klasa}
        \label{fig:dijagram_klasa}
    \end{figure}
    
\newpage
U nastavku je dat tabelarni prikaz svih entiteta sa njihovim atributima, tipovima podataka i kratkim opisom njihove uloge u sistemu.

% --- TABELA KORISNIK ---
\subsection{Tabela: Korisnik}
Klasa Korisnik predstavlja online korisnika koji kupuje i rezerviše bioskopske karte.
\vspace{5pt}

\begin{tabularx}{\textwidth}{|l|l|X|}
\hline
\textbf{Atribut} & \textbf{Tip} & \textbf{Opis} \\ \hline
email (PK) & varchar(100) & Primarni ključ, jedinstvena adresa korisnika. \\ \hline
lozinka & varchar(255) & Hash vrednost korisničke lozinke. \\ \hline
ime & varchar(50) & Ime korisnika. \\ \hline
prezime & varchar(50) & Prezime korisnika. \\ \hline
brojTelefona & varchar(20) & Kontakt telefon korisnika. \\ \hline
bonusPoeni & int & Broj sakupljenih bodova lojalnosti. \\ \hline
\end{tabularx}

% --- TABELA KARTA ---
\subsection{Tabela: Karta}
Klasa Karta sadrži podatke o karti koju je korisnik kupio. Jedan Korisnik može da kupi više karata, dok jedna karta može biti kupljena od strane jednog Korisnika.
\vspace{5pt}

\begin{tabularx}{\textwidth}{|l|l|X|}
\hline
\textbf{Atribut} & \textbf{Tip} & \textbf{Opis} \\ \hline
barkod (PK) & varchar(100) & Jedinstveni identifikator karte. \\ \hline
idKorisnika (FK) & int & Referenca na korisnika koji je kupio kartu. \\ \hline
idSedista (FK) & int & Referenca na rezervisano sedište. \\ \hline
idProjekcija (FK) & int & Referenca na specifičan termin projekcije. \\ \hline
validna & boolean & Status važnosti karte (da li je iskorišćena). \\ \hline
rezervacija & boolean & Da li je karta samo rezervisana ili i plaćena. \\ \hline
cena & int & Konačna cena karte sa uračunatim popustima. \\ \hline
\end{tabularx}

% --- TABELA PROJEKCIJA ---
\subsection{Tabela: Projekcija}
Glavna klasa sistema. Sadrži informacije o filmu koji se pušta u određenoj sali u određeno vreme kao i cenu karte. Povezana je sa tabelama: Film, Sala, Slobodna Sedišta, Bioskop
\vspace{5pt}

\begin{tabularx}{\textwidth}{|l|l|X|}
\hline
\textbf{Atribut} & \textbf{Tip} & \textbf{Opis} \\ \hline
id (PK) & varchar(50) & Jedinstveni identifikator projekcije. \\ \hline
idBioskopa (FK) & int & Referenca na bioskop u kojem se prikazuje. \\ \hline
idSale (FK) & int & Referenca na salu u kojoj se prikazuje. \\ \hline
idFilma (FK) & int & Referenca na film koji se emituje. \\ \hline
cena & float & Osnovna cena karte za ovaj termin. \\ \hline
vreme & dateTime & Tačan datum i vreme početka. \\ \hline
\end{tabularx}



% --- TABELA FILM ---
\subsection{Tabela: Film}
Klasa Film predstavlja film koji se može puštati u bioskopima.
\vspace{5pt}

\begin{tabularx}{\textwidth}{|l|l|X|}
\hline
\textbf{Atribut} & \textbf{Tip} & \textbf{Opis} \\ \hline
id (PK) & int & Identifikator filma. \\ \hline
naziv & varchar(100) & Naslov filma. \\ \hline
poster & varchar(255) & Putanja do slike postera ili ID slike. \\ \hline
trajanje & time & Dužina trajanja filma u minutima. \\ \hline
žanr & varchar(50) & Kategorija filma (npr. Akcija, Drama). \\ \hline
\end{tabularx}

% --- TABELA BIOSKOP ---
\subsection{Tabela: Bioskop}
Klasa Bioskop je jedna od glavnih klasa sistema. Referenca na nju se nalazi u klasama: Projekcija, Sala i Zaposleni.
\vspace{5pt}

\begin{tabularx}{\textwidth}{|l|l|X|}
\hline
\textbf{Atribut} & \textbf{Tip} & \textbf{Opis} \\ \hline
id (PK) & int & Jedinstveni identifikator objekta bioskopa. \\ \hline
ime & varchar(100) & Naziv bioskopa. \\ \hline
adresa & varchar(200) & Fizička adresa na kojoj se bioskop nalazi. \\ \hline
telefon & varchar(20) & Kontakt telefon objekta. \\ \hline
\end{tabularx}

% --- TABELA SALA ---
\subsection{Tabela: Sala}
Klasa Sala predstavlja salu za projektovanje filmova u jednom bioskopu.
\vspace{5pt}

\begin{tabularx}{\textwidth}{|l|l|X|}
\hline
\textbf{Atribut} & \textbf{Tip} & \textbf{Opis} \\ \hline
id (PK) & int & Jedinstveni identifikator sale. \\ \hline
idBioskopa (FK) & int & Referenca na bioskop kojem sala pripada. \\ \hline
nazivSale & varchar(50) & Naziv ili broj sale (npr. "Sala 1", "IMAX"). \\ \hline
sediste & varchar(255) & Opisni podatak o rasporedu ili kapacitetu. \\ \hline
\end{tabularx}

% --- TABELA SEDIŠTE ---
\subsection{Tabela: Sedište}
Klasa Sedište predstavlja jedno sedište u jednoj sali. Svaka sala ima više sedišta.
\vspace{5pt}

\begin{tabularx}{\textwidth}{|l|l|X|}
\hline
\textbf{Atribut} & \textbf{Tip} & \textbf{Opis} \\ \hline
idSedista (PK) & int & Jedinstveni identifikator sedišta. \\ \hline
idSale (FK) & int & Referenca na salu kojoj sedište pripada. \\ \hline
red & int & Broj reda u sali. \\ \hline
broj & int & Broj sedišta u redu. \\ \hline
tip & TipSedišta & Referenca na enumeraciju TipSedišta. \\ \hline
\end{tabularx}

% --- TABELA ZAPOSLEN ---
\subsection{Tabela: Zaposlen (Zajednička za Admin/Menadžer/Blagajnik)}
Klasa Zaposleni predstavlja zaposlenog koji koristi sistem. Konkretizacijom ove klase izvodimo klase Administrator, Menadžer i Blagajnik.

Administrator:
Klasa Administrator predstavlja Sistem Administratora koji upravlja svim podacima u bazi.

Menadžer:
Klasa Menadžer predstavlja upravnika čiji je domen uprave jedan Bioskop i sve što je sa njim povezano.

Blagajnik:
Klasa Blagajnik predstavlja zaposlenog koji prodaje karte, grickalice i pića na blagajni bioskopa korisnicima koji nisu u sistemu.
\vspace{5pt}

\begin{tabularx}{\textwidth}{|l|l|X|}
\hline
\textbf{Atribut} & \textbf{Tip} & \textbf{Opis} \\ \hline
id (PK) & int & Identifikator zaposlenog. \\ \hline
idBioskopa (FK) & int & Bioskop u kojem je zaposlen. \\ \hline
ime & varchar(50) & Ime zaposlenog. \\ \hline
prezime & varchar(50) & Prezime zaposlenog. \\ \hline
jmbg & varchar(13) & Jedinstveni matični broj građanina. \\ \hline
telefon & varchar(13) & Broj telefona. \\ \hline
email & varchar(100) & Poslovna email adresa. \\ \hline
datumZaposlenja & date & Datum zaposlenja. \\ \hline
adresa & varchar(100) & Adresa zaposlenog. \\ \hline
\end{tabularx}

% --- TABELA SLOBODNA SEDIŠTA ---
\subsection{Tabela: SlobodnaSedišta (Tabela stanja)}
Klasa Stanje Sedišta predstavlja stanje svih sedišta jedne projekcije. Vezana je za klasu Sala preko idSale. Kada korisnik kupi ili rezerviše sedište za određenu projekciju njegovo stanje se menja.
\vspace{5pt}

\begin{tabularx}{\textwidth}{|l|l|X|}
\hline
\textbf{Atribut} & \textbf{Tip} & \textbf{Opis} \\ \hline
idProjekcije (FK) & int & Projekcija za koju se proverava zauzeće. \\ \hline
idSedista (FK) & int & Sedište čije se stanje prati. \\ \hline
stanje & StanjeSedišta & Trenutni status (Slobodno, Zauzeto, itd.). \\ \hline
\end{tabularx}

\subsection{Definicije Enumeracija}
\begin{itemize}
    \item \textbf{TipSedišta}: \{STANDARDNO, LJUBAVNO, KOŽNO, VIP\}
    \item \textbf{StanjeSedišta}: \{SLOBODNO, IZABRANO, REZERVISANO, ZAUZETO\}
\end{itemize}


\section{Arhitektura sistema}

Prilikom razmatranja arhitekture informacionog sistema cilj je bio napraviti jednostavnu, široko dostupnu, stabilnu i bezbednu aplikaciju. Izborom veb aplikacije obezbeđena je široka dostupnost jer je za njeno korišćenje potrebno samo da korisnik ima internet vezu na svom računaru ili mobilnom uređaju. Pažljivo vođenje računa prilikom izrade korisničkog interfejsa postiže se jednostavnost, dok se stabilnost i pre svega bezbednost ostvaruju izborom troslojne arhitekture gde se srednji (logički) sloj deli na klijentski i serverski deo.


\begin{itemize}
    \item \textbf{Prezentacioni sloj} (Presentation Layer) 
    
    \begin{itemize}
        \item Predstavlja najviši sloj aplikacije i ima ulogu da korisniku prikaže vizuelnu reprezentaciju sadržaja na osnovu podataka koje dobija od nižeg sloja. Sastoji se od skupa HTML stranica koje su izgrađene uz pomoć HTML-a, CSS-a i JavaScript-a, a njihov detaljan izgled će biti prikazan u narednom poglavlju.
    \end{itemize}

    \item \textbf{Logički sloj} (Business Logic Layer)
    \begin{itemize}
        \item Predstavlja logički sloj servera i ima ulogu u prijemu zahteva koje klijent šalje kao i odgovaranju na iste. U okviru ovog sloja nalazi se i biznis logika celog sistema.
        Realizacija implementacije sadržaće ASP.NET Core kontrolere koji će u odnosu na zahtev dobijen od strane klijentske strane obrađivati dobijene podatke i stupati u kontakt sa slojem podataka kada to bude potrebno. Za implementaciju koristićemo Controller-Service-Repository arhitekturni obrazac.
    \end{itemize}
    \item \textbf{Sloj podataka} (Data Access Layer)
    \begin{itemize}
        \item Tabele koje će se nalaziti u ovoj bazi podataka opisane su dijagramom klasa i detaljnom specifikacijom u prethodnom poglavlju. Podaci će biti skladišteni u memoriji na serverskom računaru, i to u relacionoj bazi podataka (PostgreSQL).
    \end{itemize}
\end{itemize}


Na slici \ref{fig:dijagram_komponenti} nalazi se predlog arhitekture sistema prikazan kroz dijagram komponenti, dok slika \ref{fig:dijagram_isporucivanja} prikazuje dijagram isporučivanja koji detaljno opisuje kako su komponente raspoređene na različitim čvorovima sistema.

\begin{figure}[h]
    \centering
    \includegraphics[width=1\textwidth]{dijagram_komponenti.jpg}
    \caption{Dijagram komponenti}
    \label{fig:dijagram_komponenti}
\end{figure}

\newpage
\begin{figure}[h]
    \centering
    \includegraphics[width=1\textwidth]{dijagram_isporucivanja.jpg}
    \caption{Dijagram isporučivanja}
    \label{fig:dijagram_isporucivanja}
\end{figure}


\end{document}
