\documentclass{article}
\usepackage{graphicx}
\usepackage[serbian]{babel}

\begin{document}

\section*{Slučaj upotrebe: Kupovina na blagajni}
\label{uc:kupovina_na_blagajni}
\vspace{10pt}

\begin{itemize}
    \item \textbf{Kratak opis:} Korisnik dolazi na blagajnu bioskopa kako bi kupio kartu, preuzeo prethodno rezervisanu kartu. Nakon kupljene karte, ili u slučaju gde korisnik već poseduje kartu, blagajnik korisniku nudi kupovinu grickalica i pića. Sistem evidentira sve obavljene transakcije i ažurira podatke u bazi.
    
    \item \textbf{Akteri:} 
    \begin{itemize}
        \item \textbf{Korisnik} - osoba koja dolazi na blagajnu.
        \item \textbf{Blagajnik} - zaposleni koji obavlja prodaju i interakciju sa korisnikom.
    \end{itemize}
    \vspace{5pt}

    \item \textbf{Preduslovi:} 
    \begin{itemize}
        \item Bioskop je otvoren i blagajna je u funkciji.
        \item Korisnik je na blagajni i bar jedan blagajnik je dostupan.
        \item Sistem je povezan sa bazom podataka. 
        \item Blagajnik je ulogovan u sistem.
    \end{itemize}
    \vspace{5pt}
    
    \item \textbf{Postuslovi:} 
    \begin{itemize}
        \item Korisnik je uspešno izvršio transakciju zbog koje je došao na blagajnu, a sistem je ažurirao sve relevantne podatke u bazi.
    \end{itemize}
    \vspace{5pt}
    
    \item \textbf{Osnovni tok:}
    \begin{enumerate}
        \item Korisnik prilazi slobodnom blagajniku.
        \item Korisnik može da izabere jednu od opcija:
            \begin{itemize}
                \item Ukoliko želi da kupi kartu, prelazi se na podtok P1.
                \item Ukoliko želi da preuzme prethodno rezervisanu kartu, prelazi se na podtok P2.
                \item Ukoliko želi da kupi grickalice ili piće, prelazi se na podtok P3.
            \end{itemize}
        \item Sistem prikazuje ukupnu cenu za izabrane artikle.
        \item Blagajnik traži od korisnika način plaćanja (keš ili kartica).
        \item Korisnik bira način plaćanja i izvršava transakciju.
        \item Sistem obradjuje plaćanje.
    \end{enumerate}

    \item \textbf{Podtokovi:}
    
    P1: Kupovina karte
    \begin{enumerate}
        \item Blagajnik pita korisnika za film i termin koji želi da pogleda. 
        \item Blagajnik pregleda podatke o traženom filmu i terminu. 
        \item Sistem prikazuje dostupna sedišta za izabrani termin. 
        \item Korisnik bira sedište/a koje želi da kupi. 
        \item Blagajnik unosi izabrana sedišta u sistem. 
        \item Blagajnik unosi karte u porudžbinu. 
        \item Prelazi se na korak 2 osnovnog toka.
    \end{enumerate}

    P2: Preuzimanje rezervisane karte
    \begin{enumerate}
        \item Blagajnik traži od korisnika podatke o rezervaciji (npr. broj rezervacije, ime). 
        \item Sistem pronalazi rezervaciju i prikazuje detalje. 
        \item Blagajnik proverava validnost rezervacije.
            \begin{itemize}
                \item Ukoliko rezervacija nije validna, blagajnik obaveštava korisnika i vraća se na korak 2 osnovnog toka. 
            \end{itemize}
        \item Blagajnik potvrdjuje preuzimanje karte i unosi karte u porudžbinu. 
        \item Prelazi se na korak 2 osnovnog toka.
    \end{enumerate}

    P3: Kupovina grickalica i pića
    \begin{enumerate}
        \item Blagajnik pita korisnika koje grickalice i piće želi da kupi.
        \item Korisnik bira artikle koje želi da kupi.
        \item Blagajnik unosi izabrane artikle u porudžbinu.
        \item Prelazi se na korak 2 osnovnog toka.
    \end{enumerate}
    \vspace{5pt}
    
    \item \textbf{Alternativni tokovi:}
    \begin{itemize}
        \item \textbf{A1: Nema slobodnih sedišta:} Ukoliko nema slobodnih sedišta za izabrani termin, sistem obaveštava blagajnika koji zatim obaveštava korisnika i nudi alternativne termine ili filmove i prelazi na korak 1 podtoka P1.
        \item \textbf{A2: Korisnik odustaje od procesa:} U bilo kom momentu korisnik može odlučiti da ne nastavi sa kupovinom ili preuzimanjem karte, i blagajnik zatim završava interakciju.
        \item \textbf{A3: Neuspelo plaćanje:} Ukoliko plaćanje ne uspe, sistem obaveštava blagajnika koji zatim obaveštava korisnika i nudi opciju da pokuša ponovo ili odustane.
        \item \textbf{A4: Nedostupnost artikala (grickalice i piće):} Ukoliko izabrani artikli nisu dostupni, blagajnik obaveštava korisnika i nudi alternativne artikle ili završava proces bez kupovine artikala.
        \item \textbf{A5: Odustajanje od artikala:} Korisnik u bilo kom momentu može da odluči da ne kupi neki od artikala iz porudžbine pre nego što se predje na plaćanje, i blagajnik zatim ažurira porudžbinu u skladu sa tim.
    \end{itemize}
    \vspace{5pt}

    \item \textbf{Specijalni zahtevi:} /
    \vspace{5pt}
    
    \item \textbf{Dodatne informacije:} /
    \vspace{5pt}
\end{itemize}

\begin{figure}[h]
    \centering
    \includegraphics[width=0.9\textwidth]{dijagram_aktivnosti_kupovina_na_blagajni.jpg}
    \caption{Dijagram aktivnosti: Kupovina na blagajni}
    \label{fig:dijagram_aktivnosti_kupovina_na_blagajni}
\end{figure}

\begin{figure}[h]
    \centering
    \includegraphics[width=1\textwidth]{dijagram_sekvence_kupovina_na_blagajni.jpg}
    \caption{Dijagram sekvenci: Kupovina na blagajni}
    \label{fig:dijagram_sekvenci_kupovina_na_blagajni}
\end{figure}

\end{document}